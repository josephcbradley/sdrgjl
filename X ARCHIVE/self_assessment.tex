\documentclass[11pt]{article}
\usepackage[a4paper, portrait, margin=2cm]{geometry}


\title{Self Assessment}
\date{}
%Include a final page with your self-assessment of your project report.
%student initiative and amount of guidance required, scientific quality, breadth, presentation and logical structure of the report. 
% For each of the marking criteria mention an example of a good practice and/or a challenge you had to overcome.

\begin{document}
\maketitle
\section{Guidance Required}
Overall, I am pleased with the amount I was able to cover independently in this project. I had not studied quantum mechanics prior to this project, so my first task was to familiarise myself with the basic concepts and raise questions where necessary. The most challenging parts in terms of working without guidance were the exact solutions and understanding how to structure and present a scientific paper. For the exact solution, implementing the eigensolver took longer than I expected, and I was grateful for some code examples in a different programming language to show how it could be done. For the writing of the report, I had not written a scientific paper like this before, which led me to start writing too late and underestimate the level of detail and precision that would be needed in the introduction and review sections. Having worked very independently for most of the project I was grateful for some focused support in the last week of the project to finalise the write up. 

\section{Scientific Quality}
I was pleased with the way that I was able to investigate an extension of the original work within a small time frame, especially in an area of Physics that was new to me. My biggest challenge in terms of scientific quality was the careful summary of the complicated background material in quantum mechanics in the introduction and reviews, and in particular the reporting of results in papers that I did not have long to digest. Furthermore, it was difficult to introduce area law violations and the SDRG framework at the same time without overwhelming the text in the introduction or being too vague. In terms of the presentation, I am happy that the quality of the figures and diagrams is high, and the document is well presented and easy to navigate. If I had had longer I would have developed my knowledge of the theory of quantum systems at criticality, made the chain diagrams in the introduction neater.

\section{Breadth}
The project had a good level of focus and ambition for three months of study. A significant portion of the work involved was reproducing existing tools and results, and from there we took two small steps forward in calculating the negativity of the randbow chain and exploring the power-law systems. I enjoyed exploring the literature on some of the more distant topics (e.g. CFT, DMRG, quantum computing, Bell inequalities), even if they fell out of scope of the final report. If I had longer to work on the project, I would have liked to implement the exact solution for the negativity in the non-interacting chain and experiment further with different values of the anisotropy parameter $\Delta$, or develop the RG theory further. 


\section{Logical Structure}
Initially, I had followed the department handbook structure with a separate literature review before I discussed the measures of entanglement and the SDRG procedure. In hindsight this was a mistake, and I am grateful for my supervisor having pointed this out to me. The refactored report included the (shorter) literature reviews on entanglement and SDRG alongside the mathematical details, which was clearer and meant less repetition. The report continues with the results in a straightforward manner.

\end{document}
