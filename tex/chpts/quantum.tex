% !TEX root = ../main.tex

\section{Quantum Mechanics and the Jordan-Wigner Transformation}\label{sec:quantum}
Quantum mechanics is built on state vectors that have a different notion of state to classical state vectors. In a classical system, a state $s(t)$ at time $t$ provides all of the information needed to predict the outcome of an experiment if the experiment were conducted at time $t$. The mapping of moments in time to possible experimental outcomes is one to one. To have information about the state of a system means that you know what result you will get if you observe the system.

However, in quantum mechanics, the mapping of states to possible experimental outcomes is one to many. A quantum state does not, in general, tell us what we will observe at $t$ - rather, it encodes a range of possible observations that could be made when the experiment happens. The result of the experiment will not be known until the experiment has taken place.

In this section we will briefly summarise the quantum mechanical formalism used throughout the paper, and in particular some details of the Jordan-Wigner transformation as used in \ref{sec:randbow_old_results_exact} and \ref{subsec:powerbow_sdrg}. We will assume the reader is familiar with bra-ket notation and understands the general idea of an operator as it is used in quantum mechanics. For an excellent introduction, see \cite{cresser}.


\subsection{Commutators}\label{subsec:commutators}
Most operators do not commute, and the same is true within quantum mechanics. We define the \textit{commutator}:

\begin{equation}\label{eq:commutator}
	[A, B] = AB - BA
\end{equation}

The \textit{anti-commutator} is defined as:

\begin{equation}\label{eq:anticommutator}
	\{A, B\} = AB + BA
\end{equation}

Commutation relations are important in defining the relationships between operators. For example, given the position operators $\hat{x}$ and the momentum operator $\hat{p}$:

\begin{equation}
	[\hat{x}, \hat{p}] = i \hbar \mathbb{I}
\end{equation}

This relation between two operators that are Fourier transforms of one another is often referred to as the canonical commutation relation.


\subsection{Spin}\label{subsec:spin}
\textit{Spin} is a form of angular momentum inherent to quantum particles characterised by a \textit{spin quantum number} $s$. Restrictions on this spin quantum number imply important properties about different quantum particles. Spin is quantised and takes the form:

\begin{equation}\label{eq:spin_number}
	S = \hbar \sqrt{s(s + 1)}
\end{equation}

where $s$ can be any half-integer. Particles with half-integer spin are \textit{fermions} and particles with integer spins are called \textit{bosons}.

An important property of fermions is that they obey the Pauli Exclusion Principle \cite{pauli_nobel}. Consider a creation operator $a_i^\dagger$ that acts on a vacuum state $\ket{0}$ to create a particle at position $i$\footnote{This explanation of the exclusion principle is due to Blundell and Lancaster \cite{blundell_qft}.}. Adding a particle at position $i$ and another at position $j$ must give us the same state, up to a prefactor:

\begin{equation}
	a_i^\dagger a_j^\dagger = \lambda a_j^\dagger a_i^\dagger
\end{equation}

Restricting ourselves without loss of generality to the cases $\pm1$, we consider the bosonic case of $+1$ first, which implies that the state vector is symmetric under particle exchange. This also implies that:

\begin{gather}
	a_i^\dagger a_j^\dagger - a_j^\dagger a_i^\dagger = [a_i^\dagger, a_j^\dagger] = 0
\end{gather}

However, for the \textit{fermionic} case, we have the prefactor $-1$ and instead the anti-commutator $\{c_i^\dagger, c_j^\dagger\} = 0$, where $c_i^\dagger$ is the fermionic creation operator at $i$. Mostly importantly, if we set $i = j$ then:

\begin{gather}
	c_i^\dagger c_i^\dagger + c_i^\dagger c_i^\dagger = 0 \\
	c_i^\dagger c_i^\dagger = 0
\end{gather}

Which is exactly the Pauli Exclusion Principle: if we try to create two fermions at the same position, they annihilate and we get nothing at all. This will be relevant when we discuss the Jordan-Wigner transformation in the following section.


\subsection{The Jordan-Wigner Transformation}\label{subsec:jw_transform}
The Jordan-Wigner (JW) transformation maps a system of spins onto a system of (free) fermions. This is useful as it opens up a wider variety of technqiues for dealing with disordered, many body problems. We recommend \cite{quantum_ising_beginners} for a thorough overview, and our summary of the JW transformation relies heavily on their layout. 

Recall from \ref{subsec:spin} that a fermion obeys the Pauli exclusion principle and that one can define a creation operator $c_i^\dagger$ with anti-commutator $\{c_i^\dagger, c_j^\dagger\} = 0$,. Using this, the JW transformation maps spins to fermions according to the following transformations: 

\begin{align}
	\hat{\sigma}_j^x &=\hat{K}_j\left(\hat{c}_j^{\dagger}+\hat{c}_j\right) \\
	\hat{\sigma}_j^y &=\hat{K}_j i\left(\hat{c}_j^{\dagger}-\hat{c}_j\right) \\
	\hat{\sigma}_j^z &=1-2 \hat{n}_j
\end{align}

%\begin{equation}
%\left\{\begin{aligned}
%\hat{\sigma}_j^x &=\hat{K}_j\left(\hat{c}_j^{\dagger}+\hat{c}_j\right) \\
%\hat{\sigma}_j^y &=\hat{K}_j i\left(\hat{c}_j^{\dagger}-\hat{c}_j\right) \\
%\hat{\sigma}_j^z &=1-2 \hat{n}_j
%\end{aligned} \quad \text { with } \quad \hat{K}_j=\prod_{j^{\prime}=1}^{j-1}\left(1-2 \hat{n}_{j^{\prime}}\right) .\right.
%\end{equation}



where $\hat{n}_j=\hat{c}_j^{\dagger} \hat{c}_j$ is the fermionic number operator and $\hat{K}_j=\prod_{j^{\prime}=1}^{j-1}\left(1-2 \hat{n}_{j^{\prime}}\right)$ is a parity adjusting operator. The physical picture here is that we have gone from spins that can be `up or down' to fermions that can be present or not. 

%Roughly one can interpret the transformation for each spin operator as being something like `rather than measure the spin in such a direction at site $i$, measure the combined effect of adding and also of annihilating a fermion at $i$, adjusted for the parity condition $\hat{K}_j$'. The details are beyond the scope of this report but we use them in the calculations below for the exact solution to the $XX$ chain. 

These mappings can be reversed, following \cite{paola2016}:

\begin{equation}\label{eq:jw_to_fermions}
c_i=\left(\prod_{m=1}^{i-1} \sigma_m^z\right) \frac{\sigma_i^x-i \sigma_i^y}{2}
\end{equation}

In the following section we show how this technique can be used to solve the random inhomogeneous chain.

\subsection{Using the JW Transformation in the Exact Solution}\label{subsec:exact_sol_jw}

Again following \cite{paola2016}, we define a slightly more general Hamiltonian than equation \ref{eq:model_hamiltonian} as follows: 

\begin{equation}
H_{X X} =\sum_{i} J_i\left(S_i^x S_{i+1}^x+S_i^y S_{i+1}^y\right)+h \sum_{i} S_i^z
\end{equation}

Note that this applies strictly to the $XX$ chain, and the additional $h$ term. Using equation \ref{eq:jw_to_fermions} this is immediately transformed into a fermionic form: 

\begin{equation}
\mathcal{H}_{X X}=\frac{1}{2} \sum_{i=1}^{L-1} J_i\left(c_i^{\dagger} c_{i+1}+c_{i+1}^{\dagger} c_i\right)+\frac{h}{2} \sum_{i=1}^{L-1} c_i^{\dagger} c_i
\end{equation}

where we have defined the additional anti-commutator relation $\left\{c_m, c_n^{\dagger}\right\}=\delta_{m, n}$. 

One can now assume that each new fermion has individual eigenstates of the form:

\begin{equation}\label{eq:eta_operator_def}
\eta_q^{\dagger}|0\rangle = \sum_i \Phi_q(i) c_i^{\dagger}|0\rangle
\end{equation}

where $q$ labels the different eigenstates and $\Phi$ is a vector of amplitudes to be found. The Schrödinger equation becomes: 

\begin{equation}
J_i \Phi_q(i+1)+J_{i-1} \Phi_q(i-1)=2 \epsilon_q \Phi_q(i)
\end{equation}

where $\epsilon_q$ are single particle eigenvalues per eigenstates $q$. This is a new eigenvalue problem for a banded $2L \times 2L$ matrix with the couplings $J_i$ on the off-diagonals. 

The groundstate of the original Hamiltonian will have $L \div 2$ fermions, giving us the following: 

\begin{equation}
|G S\rangle=\eta_{q_M}^{\dagger} \eta_{q_{M-1}}^{\dagger} \cdots \eta_{q_1}^{\dagger}|0\rangle
\end{equation}

Multiplying equation \ref{eq:eta_operator_def} by the relevant operators, we can derive the anti-commutators:

\begin{equation}
\left\{\eta_q^{\dagger}, c_j^{\dagger}\right\}=\left\{\eta_q, c_j\right\}=0
\end{equation}

and

\begin{equation}
\left\{\eta_q^{\dagger}, c_j\right\}=\Phi_q(j) \delta_{k, j}, \quad\left\{\eta_q, c_j^{\dagger}\right\}=\Phi^*(j) \delta_{k, j}
\end{equation}

Combining the previous two equations, we can derive the two point function's expected value for the groundstate of the original Hamiltonian:

\begin{equation}
\left\langle c_i^{\dagger} c_j\right\rangle=\sum_q \Phi_q^*(i) \Phi_q(j)
\end{equation}

This defines a matrix $C_{i, j}$ that can be restricted to some subsystem $A$ for the purposes of analysing a given disorder realisation. In particular, from \cite{peschelReducedDensityMatrices2009}, we can calculate the entanglement entropy:

\begin{equation}
S_A=-\sum_l\left(\lambda_k \ln \lambda_k+\left(1-\lambda_k\right) \ln \left(1-\lambda_k\right)\right)
\end{equation}

This completes our overview of the technique used in \ref{sec:randbow_old_results_exact} to calculate the entanglement entropy exactly.
%\begin{equation}\label{eq:boson_creator}
%c_{i}=\left(\prod_{m=1}^{i-1} \sigma_{m}^{z}\right) \frac{\sigma_{i}^{x}-i \sigma_{i}^{y}}{2}
%\end{equation}



