% !TEX root = ../main.tex
%A number of results for the scaling of entanglement measures already exist (see \cite{refael2004}, \cite{paola2016}, \cite{paola2018}). 
\section{Disordered Chain: Existing Results}\label{sec:disordered_old_results}
In this section we will reproduce the results of \cite{paola2016} for the disordered spin chain and in the process verify that our SDRG algorithm and related code works as expected.

\subsection{Entanglement Entropy: Analytic Results and Numerical SDRG}\label{subsec:baseline_entropy}
As per \cite{paola2016}, we measure the entanglement entropy of the disordered chain with and without periodic boundary conditions. We start be reviewing the analytic expectations for the disordered spin chain with the SDRG method. 

We observe that, given the RSP state in equation \ref{eq:groundstate}, to calculate the entanglement entropy we only need to know the number of singlets in $A$ that connect to the remainder of the chain $B = A^\prime$. In the basis $\ket{\uparrow} = \begin{bmatrix}
	1 \\
	0 
\end{bmatrix}$ and $\ket	{\downarrow} = \begin{bmatrix}
	0 \\
	1 
\end{bmatrix}$ we can make the singlet state $\ket{s}$ (see equation \ref{eq:micro_groundstate}) explicit. The two spin vectors become: 

\begin{equation}
	\ket{\uparrow \downarrow} = \begin{bmatrix}
	0 \\
	1 \\
	0 \\
	0 \\
\end{bmatrix}, \ket{\downarrow \uparrow} = \begin{bmatrix}
	0 \\
	0 \\
	1 \\
	0 \\
\end{bmatrix}
\end{equation}

Recalling the singlet state vector (equation \ref{eq:micro_groundstate}), the singlet density matrix is then:

\begin{equation}\label{eq:micro_groundstate_explicit}
	\rho_{2S} = \ket{s}\bra{s} = \frac{1}{2}\left(\begin{array}{cccc}
0 & 0 & 0 & 0 \\
0 & 1 & -1 & 0 \\
0 & -1 & 1 & 0 \\
0 & 0 & 0 & 0
\end{array}\right)
\end{equation}

Taking the partial trace over the second spin, we have:

\begin{equation}
\rho_S=\frac{1}{2}\left(\begin{array}{ll}
1 & 0 \\
0 & 1
\end{array}\right)
\end{equation}

To calculate the entanglement entropy, we note that $\rho_S$ is Hermitian and use the spectral theorem on equation \ref{eq:neumann_entropy} to get:

\begin{equation}
	S = - \sum_i \lambda_i \ln{\lambda_i} = \ln{2}
\end{equation}

where $\{\lambda_i\}$ are the eigenvalues of $\rho_S$. At this point, the RSP structure of the grounstate becomes very useful. Following \cite{refael2004} we can say that the entanglement entropy is just `the number of singlets that connect sites inside to sites outside the segment', multiplied by the entropy of each such segment, $\ln{2}$. This is a very simple expression of the entropy for a complex many-body system. We follow \cite{paola2016} and let the number of singlets connecting subsystem $X$ to $Y$ be $n_{X:Y}$, so that the entropy of subsystem $A$ is given by:

\begin{equation}
	S_A = \ln{2} \times n_{A: B}
\end{equation}

This is consistent up to a constant with the SDRG result as reported in \cite{paola2016} and \cite{Laflorencie2018}:

\begin{equation}
S_{A}=\frac{\ln 2}{3} \ln \ell+K
\end{equation}

which implies that, taking averages over disorder, $\langle n_{A: B} \rangle = \frac{\ln 2}{3}$. 
%The position in the chain is not as relevant as it is in the later models where a strong spatial inhomogeneity is introduced, because the non-locality is much weaker and is not focused around a particular point (e.g. the centre). 
%It is known from \cite{refael2004} that the ground state of the random spin chain (equation \ref{eq:model_hamiltonian}) is an RSP, made up of $L \div 2$ singlets each entangled in a valence bond, e.g. a valence bond state. Each pair of singlets has the state given in equation \ref{eq:micro_groundstate}.

To verify this numerically, we run the SDRG procedure on the $XXX$ chain: for each simulation, we draw $L$ random couplings from the uniform distribution over $[0, 1]$ and run the SDRG algorithm. This returns a vector of $L \div 2$ singlets.  On this vector of singlet pairs we  calculate the entanglement entropy for every realisation of the RSP for all window sizes $l$, and maintain a running mean of the result. 
%The entanglement entropy is calculated by dividing the system into the subsystem $A$ and its complement $A'$: the entropy is the number of singlet links between these two subsections multiplied by $\ln 2$.  

%(figures \ref{fig:baseline_entropy_pbc} and \ref{fig:baseline_entropy_pbc_raw})
%In this simulation we measure the entanglement entropy of a subsystem of length $l$ located in the left hand side of the the XXX chain. 
%For the open case (figures \ref{fig:baseline_entropy_obc} and \ref{fig:baseline_entropy_obc_raw}) the need is especially acute. 


 We run the analysis for $L = 1000$ and $L = 2000$ for $50,000$ disorder realisations and in each case calculate the entanglement entropy from $l = 10$ to $l = L \div 2$ with an interval of $10$ in between. Our results are shown in figure \ref{fig:baseline_entanglement}, with fitted curves for the adjusted $l$. As can be seen in all four subfigures of figure \ref{fig:baseline_entanglement}, entropy scales logarithmically with the subsystem size. This is clearly an area law violation, and perfectly follows the prediction in \cite{refael2004}.
 
These analytic predictions are accurate for $l \leq L \div 2$, after which finite size effects begin to dominate. For the periodic case the finite size effects are smaller, and can be corrected with the following two maps: 

\begin{equation}\label{eq:L_mapping}
\ell \rightarrow L_{c} \equiv \frac{L}{\pi} Y\left(\frac{\pi \ell}{L}\right)
\end{equation}

\begin{equation}
Y(x)=\sin (x)\left(1+\frac{4}{3} k_{1} \sin ^{2}(x)\right)
\end{equation}

where $k_1 = 0.115$, given by \cite{fagotti_2011}. 
% In the unadjusted $l$ cases we notice that for large $l$ the entropy begins to decrease again, as we would expect in a finite system.
 % It should be noted that in figure \ref{fig:baseline_entropy_pbc_raw} the degree of divergence from the log-linear trend is less than is seen in \cite{paola2016}. This is because we have chosen to calculate the entanglement entropy only up to $L \div 2$. % as beyond half the chain, finite size effects will emerge even for the periodic implementation.

\begin{figure}
     \centering
     \begin{subfigure}[b]{0.49\textwidth}
    \centering
  \includegraphics[width=\textwidth]{../data/imgs/entropy_baseline_pbc}
    \caption{Entanglement entropy of the $XXX$ chain, adjusted $l$, PBC}
    \label{fig:baseline_entropy_pbc}
\end{subfigure}
     \hfill
     \begin{subfigure}[b]{0.49\textwidth}
         \centering
    \includegraphics[width=\textwidth]{../data/imgs/entropy_baseline_obc}
    \caption{Entanglement entropy of the $XXX$ chain, adjusted $l$, OBC}
    \label{fig:baseline_entropy_obc}
     \end{subfigure}
  	\begin{subfigure}[b]{0.49\textwidth}
    \centering
  \includegraphics[width=\textwidth]{../data/imgs/entropy_baseline_pbc_raw}
    \caption{Entanglement entropy of the $XXX$ chain, no $l$ adjustment, PBC}
    \label{fig:baseline_entropy_pbc_raw}
\end{subfigure}
\hfill
\begin{subfigure}[b]{0.49\textwidth}
    \centering
  \includegraphics[width=\textwidth]{../data/imgs/entropy_baseline_obc_raw}
    \caption{Entanglement entropy of the $XXX$ chain, no $l$ adjustment, OBC}
    \label{fig:baseline_entropy_obc_raw}
\end{subfigure}
     \caption{Entanglement entropy, recalculated from \cite{paola2016}. We measure the entanglement entropy of a subsystem of length $l$ located in the left hand side of the the XXX chain. Each simulation is run for $50,000$ disorder realisations. \ref{fig:baseline_entropy_pbc}: the periodic chain with the adjusted subsystem length $L_c$. \ref{fig:baseline_entropy_obc}: the open chain with the adjusted subsystem length $L_c$. \ref{fig:baseline_entropy_pbc_raw}: the periodic chain with the unadjusted subsystem length $l$. \ref{fig:baseline_entropy_obc_raw}: the open chain with the unadjusted subsystem length $l$.}
        \label{fig:baseline_entanglement}
\end{figure}


\subsection{Logarithmic Negativity: Analytic Results and Numerical SDRG}\label{subsec:baseline_negativity}
In this section we recalculate the logarithmic negativity as reported in \cite{paola2016}. We will start by reviewing the analytic prediction via the SDRG method. We note that the groundstate as described in equation \ref{eq:groundstate} is a tensor product of singlet states, and that from section \ref{subsec:entropy_negativity} the logarithmic negativity is additive under the tensor product. Before we go further it is useful to note that the partial transpose $\rho_{2 S}^{T_2}$ of $\rho_{2S}$ with respect to $A_2$ is:

\begin{equation}
\rho_{2 S}^{T_2}=\frac{1}{2}\left(\begin{array}{cccc}
0 & 0 & 0 & -1 \\
0 & 1 & 0 & 0 \\
0 & 0 & 1 & 0 \\
-1 & 0 & 0 & 0
\end{array}\right)
\end{equation}
%This is achieved in this simple case by transposing each $2 \times 2$ block of the matrix in equation \ref{eq:micro_groundstate_explicit}.
Furthermore, distinguishing between singlets which connect $A$ to $A$, $B$ to $B$, and $A$ to $B$, we can follow \cite{paola2016} and describe the complete RSP as:

\begin{equation}
\rho_{R S P}=\bigotimes_{i=1}^{n_{A: A}} \rho_{2 S} \bigotimes_{i=1}^{n_{B: B}} \rho_{2 S} \bigotimes_{i=1}^{n_{A: B}} \rho_{2 S}
\end{equation}

We then trace over the $B$ subsystem, leaving us with:

\begin{equation}
\rho_A=\bigotimes_{i=1}^{n_{A: A}} \rho_{2 S} \bigotimes_{i=1}^{n_{A: B}} \rho_S
\end{equation}

and after taking the partial transpose:

\begin{equation}
\rho_A^{T_2}=\bigotimes_{i=1}^{n_{A: A}} \rho_{2 S}^{T_2} \bigotimes_{i=1}^{n_{A: B}} \rho_S^{T_2}
\end{equation}

The logarithmic negativity is additive over this, and according to \cite{paola2016} it simplifies to:

\begin{equation}
\mathcal{E}_{A_1: A_2}=n_{A_1: A_2} \ln \operatorname{Tr}\left|\rho^{T_2}_{2S}\left(A_1 \cup A_2\right)\right|
\end{equation}

where $\rho$ is the density matrix of the generic singlet state. As discussed in section \ref{subsec:entropy_negativity}, the trace norm of a Hermitian matrix is just the sum of the absolute value of its eigenvalues. The eigenvalues of $\rho^{T_2}_{2S}$ are $\{-1/2, 1/2, 1/2, 1/2\}$, thus the trace norm is $2$ and the logarithmic negativity of the total subsystem $A$ is:

\begin{equation}
\mathcal{E}_{A_1: A_2}=n_{A_1: A_2} \ln 2
\end{equation}

This simply says that the logarithmic negativity of the subsystem is $\ln{2}$ multiplied by the number of singlets shared between $A_1$ and $A_2$.

This result allows us to calculate the logarithmic negativity over one disorder realisation. Taking overages over the disorder with $\langle \cdot \rangle$, we have:

\begin{equation}\label{eq:disordered_logneg_n}
	\langle \mathcal{E}_{A_1: A_2} \rangle = \langle n_{A_1: A_2} \rangle \ln 2
\end{equation}

In \cite{paola2016} an SDRG result for $\langle n_{A_1: A_2} \rangle$ is also given for two adjacent intervals of lengths $l_1, l_2$ as:

\begin{equation}\label{eq:disordered_n}
\left\langle n_{A_1: A_2}\right\rangle=\frac{1}{6} \ln \left(\frac{\ell_1 \ell_2}{\ell_1+\ell_2}\right)
\end{equation}

with results for disjoint systems also available. Combining equations \ref{eq:disordered_logneg_n} and \ref{eq:disordered_n} we get the final results for the logarithmic negativity: 

\begin{equation}\label{eq:disordered_logneg}
	\langle \mathcal{E}_{A_1: A_2} \rangle = \frac{\ln 2}{6} \ln \left(\frac{\ell_1 \ell_2}{\ell_1+\ell_2}\right)
\end{equation}


To verify this numerically, we calculate the logarithmic negativity of the one dimensional XXX chain as in \cite{paola2016} via the numerical SDRG method. Each subsystem is of length $l$ and are extended in increments of $10$ for every disorder realisation. The simulations are for the adjoint case, i.e. $r = 0$. We run our analysis on systems of $L = 1000$ and $L = 2000$ for $50, 000$ disorder realisations in the OBC and PBC cases. The results can be seen in figure \ref{fig:baseline_negativity}. In particular we have plotted the shifted negativity as defined in \cite{paola2016}:

\begin{equation}\label{eq:shifted negativity}
\mathcal{E}_{A_{1}: A_{2}}^{s}=\mathcal{E}_{A_{1}: A_{2}}-\frac{\ln 2}{6} \ln L
\end{equation}

such that simulations for different $L$ collapse onto the same curve. Furthermore, for the periodic case we can fit a more accurate adjusted curve as reported in \cite{paola2018}:

\begin{equation}
\mathcal{E}_{A_1: A_2}^s \simeq \frac{\ln 2}{6} \ln \frac{Y_c^2(\pi \ell / L)}{Y_c(2 \pi \ell / L)}+k
\end{equation}

which is an excellent fit for the periodic chain. For the open chain, we use equation \ref{eq:disordered_logneg}, which is a good fit for $\ell/L << 1$. 

%In both cases (entanglement entropy and logarithmic negativity), it can be seen that the entanglement scales with a logarithmic correction to the area law for $l < L/2$. Beyond this point the finite size effects begin to dominate and we see the logarithmic negativity increase sharply, in line with what is reported in \cite{paola2018}. 

\begin{figure}
     \centering
     \begin{subfigure}[b]{0.49\textwidth}
   \centering
    \includegraphics[width=\textwidth]{../data/imgs/baseline_negativity_plot_pbc}
    \caption{Shifted logarithmic negativity as a function of $l/L$ for the for $XXX$ chain, $\delta = 1$, periodic boundary conditions.}
    \label{fig:baseline_negativity_pbc}
\end{subfigure}%
     \hfill
     \begin{subfigure}[b]{0.49\textwidth}
         \centering
    \includegraphics[width=\textwidth]{../data/imgs/baseline_negativity_plot_obc}
    \caption{Shifted logarithmic negativity as a function of $l/L$ for the for $XXX$ chain, $\delta = 1$, open boundary conditions.}
    \label{fig:baseline_negativity_obc}
     \end{subfigure}
            \caption{Shifted logarithmic negativity, recalculated from \cite{paola2016}. In both figures, we measure the logarithmic negativity of two adjacent subsystems of length $l$ in the XXX chain. Each simulation is run for $50,000$ trials. For implementation details, see section \ref{subsec:baseline_negativity}. \ref{fig:baseline_negativity_pbc}: shifted logarithmic negativity of the periodic chain with the subsystem length $l$. \ref{fig:baseline_negativity_obc}: shifted logarithmic negativity of the open chain with the subsystem length $l$.}
        \label{fig:baseline_negativity}
\end{figure}

To summarise the preceding sections, we have verified that for disordered models, both entanglement entropy and logarithmic negativity scale logarithmically. In the following section we will review the strongly inhomogeneous `randbow chain'. 